\section{Dominik Motyl}
\label{sec:Dominik Motyl}

\subsection{Twierdzenie o trzech ciągach}
Niech $ a_n, b_n, c_n \textrm{ będą ciągami liczbowymi takimi że } $
\[ \displaystyle{\lim_{n \to \infty}} a_n = \displaystyle{\lim_{n \to \infty}} c_n = g \] oraz \\
\[ 
\exists n_0 \in \mathbb{N} : \forall n_0 \ a_n \leq b_n \leq c_n 
\] wówczas  
\[ \lim_{n \to \infty} b_n = g \]

\subsection{O czym zdarza mi się myśleć podczas robienia zadań z granic}


Poniżej znajduje się obrazek z bardzo grzecznym piesełem (Figure~\ref{fig:słynny pieseł}).

\begin{figure}[htbp]
    \centering
    \includegraphics[width=0.7\textwidth]{pictures/piesel.jpg}
    \caption{Tak zwany, Pieseł, szerzej znany ze względu na liczne memy.}
    \label{fig:słynny pieseł}
\end{figure}

\subsection{Szachownica Polibiusza}

Poniższa tabela to tak zwana Szachownica Polibiusza w wersji dla alfabetu łacińskiego \ref{tab:Szachownica Polibiusza}

\begin{table}[htpb]
    \centering
    \begin{tabular}{|
        >{\columncolor[HTML]{9B9B9B}}c |
        >{\columncolor[HTML]{F8F9FA}}l |
        >{\columncolor[HTML]{F8F9FA}}l |
        >{\columncolor[HTML]{F8F9FA}}l |
        >{\columncolor[HTML]{F8F9FA}}l |
        >{\columncolor[HTML]{F8F9FA}}l |}
        \hline
         &
          \multicolumn{1}{c|}{\cellcolor[HTML]{9B9B9B}{\color[HTML]{000000} \textbf{1}}} &
          \multicolumn{1}{c|}{\cellcolor[HTML]{9B9B9B}{\color[HTML]{000000} \textbf{2}}} &
          \multicolumn{1}{c|}{\cellcolor[HTML]{9B9B9B}{\color[HTML]{000000} \textbf{3}}} &
          \multicolumn{1}{c|}{\cellcolor[HTML]{9B9B9B}{\color[HTML]{000000} \textbf{4}}} &
          \multicolumn{1}{c|}{\cellcolor[HTML]{9B9B9B}{\color[HTML]{000000} \textbf{5}}} \\ \hline
        \textbf{1} & {\color[HTML]{202122} A} & {\color[HTML]{202122} B} & {\color[HTML]{202122} C} & {\color[HTML]{202122} D}   & {\color[HTML]{202122} E} \\ \hline
        \textbf{2} & {\color[HTML]{202122} F} & {\color[HTML]{202122} G} & {\color[HTML]{202122} H} & {\color[HTML]{202122} I/J} & {\color[HTML]{202122} K} \\ \hline
        \textbf{3} & {\color[HTML]{202122} L} & {\color[HTML]{202122} M} & {\color[HTML]{202122} N} & {\color[HTML]{202122} O}   & {\color[HTML]{202122} P} \\ \hline
        \textbf{4} & {\color[HTML]{202122} Q} & {\color[HTML]{202122} R} & {\color[HTML]{202122} S} & {\color[HTML]{202122} T}   & {\color[HTML]{202122} U} \\ \hline
        \textbf{5} & {\color[HTML]{202122} V} & {\color[HTML]{202122} W} & {\color[HTML]{202122} X} & {\color[HTML]{202122} Y}   & {\color[HTML]{202122} Z} \\ \hline
    \end{tabular}
    \label{tab:Szachownica Polibiusza}
\end{table}

\subsection{Listy i im podobne}

\subsubsection{Listy zagnieżdżone}

\begin{enumerate}
    \item Pierwszy
        \begin{enumerate}
            \item Podpunkt 1.1
                
            \item Podpunkt 1.2
        \end{enumerate}
    \item Drugi
        \begin{enumerate}
            \item Podpunkt 2.1
                
            \item Podpunkt 2.2
        \end{enumerate}
    \item Trzeci
        \begin{enumerate}
            \item Podpunkt 3.1
                
            \item Podpunkt 3.2
        \end{enumerate}
\end{enumerate}

\subsubsection*{Jaka nazwa pliku jest najlepsza?}

	\begin{tasks}(1)
		\task gra-grupa3.exe 
		\task AstroPhase.exe 
		\task freeIphone25s\_not\_trojan\_for\_sure.exe
		\task AwsomePythonGamexe.pdf 
	\end{tasks}

\subsection{bigos}


W kociołkach \textbf{bigos} grzano; w słowach wydać trudno
\textbf{bigos} u \textit{smak przedziwny, kolor i woń cudną};
Słów tylko brzęk usłyszy i rymów porządek,
\textit{Ale treści ich miejski \underline{nie pojmie} żołądek}.
Aby cenić litewskie pieśni i potrawy,
Trzeba mieć zdrowie, na wsi żyć, wracać z obławy.
Przecież i bez tych przypraw potrawą nie lada
Jest \textbf{bigos} , bo się z jarzyn dobrych sztucznie składa.
Bierze się doń siekana, kwaszona kapusta,
Która, wedle przysłowia, \textit{sama idzie w usta};

Zamknięta w kotle, łonem wilgotnym okrywa
Wyszukanego cząstki najlepsze mięsiwa;
I praży się, aż ogień wszystkie z niej wyciśnie
\textit{Soki żywne, aż z brzegów naczynia war pryśnie
I powietrze dokoła zionie aromatem}.
\textbf{bigos}  już gotów. \underline{Strzelcy z trzykrotnym wiwatem},
Zbrojni łyżkami, biegą i bodą naczynie,
Miedź grzmi, dym bucha, \textbf{bigos}  jak kamfora ginie,
Zniknął, uleciał; tylko w czeluściach saganów
Wre para, jak w kraterze zagasłych wulkanów.

