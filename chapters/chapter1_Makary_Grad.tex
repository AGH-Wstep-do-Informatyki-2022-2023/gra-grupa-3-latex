\section{Makary Grad}
\label{sec:makary_grad}

Oto wzór Eulera: \[e^{i\pi}+1=0\]

Poniżej znak graficzny Uczelni (Figure~\ref{fig:agh}).

\begin{figure}[htbp]
    \centering
    \includegraphics[width=0.3\textwidth]{pictures/agh.jpg}
    \caption{Znak graficzny AGH.}
    \label{fig:agh}
\end{figure}

Tabela~\ref{tab:wartosci_funkcji_trygonometrycznych} przedstawia wybrane wartości funkcji trygonometrycznych.

\begin{table}[htbp]
\centering
\begin{tabular}{|c|c|c|c|c|c|c|}
\hline
$\alpha$    & 0 & $\frac{\pi}{6}$      & $\frac{\pi}{4}$      & $\frac{\pi}{3}$      & $\frac{\pi}{2}$ & $\pi$ \\ \hline
$sin\alpha$ & 0 & $\frac{1}{2}$        & $\frac{\sqrt{2}}{2}$ & $\frac{\sqrt{3}}{2}$ & 1               & 0     \\ \hline
$cos\alpha$ & 1 & $\frac{\sqrt{3}}{2}$ & $\frac{\sqrt{2}}{2}$ & $\frac{1}{2}$        & 0               & -1    \\ \hline
$tg\alpha$  & 0 & $\frac{\sqrt{3}}{3}$ & 1                    & $\sqrt{3}$           & -               & 0     \\ \hline
$ctg\alpha$ & - & $\sqrt{3}$           & 1                    & $\frac{\sqrt{3}}{3}$ & 0               & -     \\ \hline
\end{tabular}
\label{tab:wartosci_funkcji_trygonometrycznych}
\caption{Wybrane wartości funkcji trygonometrycznych.}
\end{table}

Data może być zapisana na kilka sposobów, oto przykłady:
\begin{itemize}
  \item 1.12.2022
  \item[-] 1 XII 2022
  \item[!] 1 grudnia 2022
\end{itemize}

Oto 3 różne sposoby zapisu daty:
\begin{enumerate}
  \item 1.12.2022
  \item 1 XII 2022
  \item 1 grudnia 2022
\end{enumerate}


Krótki tekst (2 akapity):

\textbf{Lorem ipsum} dolor sit amet, consectetur adipiscing elit, sed do eiusmod tempor incididunt ut labore et dolore magna aliqua. Ut enim ad minim veniam, quis nostrud exercitation ullamco laboris nisi ut aliquip ex ea commodo consequat. \par \underline{Duis aute irure dolor in reprehenderit} in voluptate velit esse cillum dolore eu fugiat nulla pariatur. Excepteur sint occaecat cupidatat non proident, sunt in culpa qui officia deserunt mollit anim id \emph{est laborum}.
